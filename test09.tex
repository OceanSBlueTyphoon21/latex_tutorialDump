\documentclass[11pt,letterpaper]{article}
\usepackage[utf8]{inputenc}
\usepackage{amsmath}
\usepackage{amsfonts}
\usepackage{amssymb}
\usepackage[margin=0.75in, paperwidth=8.5in, paperheight=11in]{geometry}

\author{Anthony Bruno}
\title{Calculus Notation}

\begin{document}
\maketitle

The function $f(x)=(x-3)^2+\dfrac{1}{2}$ has a domain $\mathrm{D}_f:(-\infty,\infty)$ and range $\mathrm{R}_f:\left[\dfrac{1}{2}, \infty\right)$. \\

$\lim\limits_{x \to a^-} f(x)$\\

$\displaystyle{\lim\limits_{x \to a} \dfrac{f(x)-f(a)}{x-a}=f'(a)}$\\

$\displaystyle{\int \sin(x)\,\mathrm{d}x=-\cos(x)+C}$\\

$\displaystyle{\int \limits_a^b}$\\

$\displaystyle{\int_{2a}^{5b}}$\\

$\displaystyle{\int_{a}^{b}x^2\,\mathrm{d}x=\left[\dfrac{x^3}{3}\right]_{a}^{b}=\dfrac{b^3}{3}-\dfrac{a^3}{3}}$\\

$\displaystyle{\sum \limits_{n=1}^{\infty}ar^n=a+ar+ar^2+\cdots+ar^n}$\\

% Riemann Sum Definition

$$\displaystyle{\int_a^b f(x)\,\mathrm{d}x = \lim \limits_{x \to \infty} \sum \limits_{k=1}^{n} f(x_k) \cdot \Delta x}$$\\

$$\vec{v}=v_1 \vec{i} + v_2 \vec{j} = \langle v_1,v_2 \rangle$$

\end{document}