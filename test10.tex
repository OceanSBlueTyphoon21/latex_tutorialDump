\documentclass[11pt]{article}

\usepackage[margin=1in]{geometry}
\usepackage{amsfonts,amsmath,amssymb}
\usepackage[none]{hyphenat}

\usepackage{graphicx}
\usepackage{float}

\usepackage[nottoc, notlot, notlof]{tocbibind}

\usepackage{fancyhdr}

\pagestyle{fancy}
\fancyhead{}
\fancyfoot{}
\fancyhead[L]{\slshape\MakeUppercase{Place Title Here}}
\fancyhead[R]{\slshape Student Name}
\fancyfoot[C]{\thepage}
%\renewcommand{\headrulewidth}{0pt}
\renewcommand{\footrulewidth}{0pt}

\parindent 0ex




\begin{document}

\begin{titlepage}
\begin{center}
\vspace{1cm}
\Large{\textbf{IB Mathematics SL}}\\
\Large{\textbf{Internal Assessment}}\\
\vfill
\line(1,0){400}\\[1mm]
\huge{\textbf{This is a Sample Title}}\\[3mm]
\Large{\textbf{- This is a Sample Subtitle -}}\\[1mm]
\line(1,0){400}
\vfill
by Student\\
Candidate \# \\
\today \\
\end{center}
\end{titlepage}

\tableofcontents
\thispagestyle{empty}
\clearpage
\setcounter{page}{1}

\section{introduction}
It will enable students to acquire attributes of the IB learner profile\footnote{An example footnote}.

\section{Scoring Criteria}

\subsection{Communication}

\subsection{Mathematical Presentation}

\subsection{Personal Engagement}

\subsection{Reflection}

\subsection{Use of Mathematics}

\section{Conclusion}

\section{Using \LaTeX}
Some text about \LaTeX.


\end{document}